\begin{section}{Lifetimes}

  \begin{frame}
    \centerline{
      \huge{Lifetimes}
    }
  \end{frame}

  \begin{frame}[fragile]
    \frametitle{What do you expect this to do? \cite{RustLang}}
    \begin{minted}{rust}
      pub fn main() {
        let mut s1: &str = "hello";
        let mut s2: &str = "world";
        let result = do_something(&s1, &s2);
        println!("{}", result);
      }

      fn do_something(s1: &str, s2: &str) -> &str {
        //do something
        s1
      }
    \end{minted}
  \end{frame}


  \begin{frame}[fragile]
    \frametitle{Rust Befuddles Us}
    \begin{minted}[escapeinside=??]{rust}
      pub fn main() {
        let mut s1: &str = "hello";
        let mut s2: &str = "world";
        let result = do_something(&s1, &s2);
        println!("{}", result);
      }

      fn do_something(s1: &str, s2: &str) -> &str ?\tikzmark{lifetime_error}? {
        //do something
        s1
      }
    \end{minted}
    \begin{tikzpicture}[overlay,remember picture]
      \draw[thick,->] (pic cs:lifetime_error) ++(-0.5,-0.1) to[bend left]
      ++(-0.5,-1.5)node[left]{\scriptsize Rust doesn't know lifetime of returned string};
    \end{tikzpicture}
  \end{frame}

  \begin{frame}[fragile]
    \frametitle{The `Fix`}
    \begin{minted}[escapeinside=??]{rust}
      pub fn main() {
        let mut s1: &str = "hello";
        let mut s2: &str = "world";
        let result = do_something(&s1, &s2);
        println!("{}", result);
      }

      fn do_something<'a>(s1: &'a str, s2: &'a str) -> &'a str ?\tikzmark{lifetime_fix}? {
        //do something
        s1
      }
    \end{minted}
    \begin{tikzpicture}[overlay,remember picture]
      \draw[thick,->] (pic cs:lifetime_fix) ++(-0.5,-0.1) to[bend left]
      ++(-2,-1.5)node[left]{\scriptsize We say all Strings have the same life};
    \end{tikzpicture}
  \end{frame}

    \begin{frame}[fragile]
    \frametitle{Why are Lifetimes necessary? \cite{RustLang}}
    \begin{minted}[escapeinside=??]{rust}
      pub fn main() {
        let mut s1 = String::from("hello");
        let result;
        {
          let mut s2 = String::from("world");
          result = do_something(&s1, &s2);
        }
        println("{}", result);
      }

      fn do_something<'a>(s1: &'a str, s2: &'a str) -> &'a str {
        //do something
        s1
      }
    \end{minted}
  \end{frame}

  \begin{frame}[fragile]
    \frametitle{Why are Lifetimes necessary?}
    \begin{minted}[escapeinside=??]{rust}
      pub fn main() {
        let mut s1 = String::from("hello");
        let result;
        {
          let mut s2 = String::from("world");
          result = do_something(&s1, &s2);
        } ?\tikzmark{s2_lifetime_error}?
        println("{}", result);
      } ?\tikzmark{s1_lifetime_error}?

      fn do_something<'a>(s1: &'a str, s2: &'a str) -> &'a str {
        //do something
        s1
      }
    \end{minted}
    \begin{tikzpicture}[overlay,remember picture]
      \draw[thick,->] (pic cs:s2_lifetime_error) ++(0,-0.05) --
      ++(5.5,0)node[right]{\scriptsize \mintinline{rust}{s2} is only valid till here};
    \end{tikzpicture}
    \begin{tikzpicture}[overlay,remember picture]
      \draw[thick,->] (pic cs:s1_lifetime_error) ++(0,-0.05) --
      ++(5.5,0)node[right]{\scriptsize \mintinline{rust}{s1} is valid till here};
    \end{tikzpicture}
  \end{frame}

\end{section}
