\begin{section}{Ownership and Borrowing}

  \begin{frame}
    \centerline{
      \huge{Ownership and Borrowing}
    }
  \end{frame}

  \begin{frame}[fragile]
    \frametitle{A Simple Program}
    \begin{minted}{rust}
      pub fn main() {
        let v = vec!(1,2,3);
        println!("{:?}", v);
      }
    \end{minted}
  \end{frame}

  \begin{frame}[fragile]
    \frametitle{What the Rust compiler does?}
    \begin{minted}[escapeinside=??]{rust}
      pub fn main() {
        let v = vec!(1,2,3); ?\tikzmark{compiler_allocates}?
        println!("{:?}", v);
      } ?\tikzmark{compiler_free}?
    \end{minted}
    \begin{tikzpicture}[overlay,remember picture]
      \draw[thick,->] (pic cs:compiler_allocates) ++(0,0.1) --
      ++(1,0)node[right]{\scriptsize Rust compiler allocates memory in heap for \mintinline{rust}{v} here};
    \end{tikzpicture}
    \begin{tikzpicture}[overlay,remember picture]
      \draw[thick,->] (pic cs:compiler_free) ++(0,0.1) --
      ++(4,0)node[right]{\scriptsize Rust compiler drops the vector \mintinline{rust}{v} here};
    \end{tikzpicture}
  \end{frame}

  \begin{frame}[fragile]
    \frametitle{Another Simple Program}
    \begin{minted}{rust}
      pub fn main() {
        let v = vec!(1,2,3);
        do_something(v);
        println!("{:?}", v);
      }

      fn do_something(v: Vec<u64>) {
        // Do something with v
      }
    \end{minted}
  \end{frame}

  \begin{frame}[fragile]
    \frametitle{What happens here?}
    \begin{minted}[escapeinside=??]{rust}
      pub fn main() {
        let v = vec!(1,2,3);
        do_something(v); ?\tikzmark{ownership_moved}?
        println!("{:?}", v); ?\tikzmark{ownership_compiler_error}?
      }

      fn do_something(v: Vec<u64>) {
        // Do something with v
      } ?\tikzmark{ownership_compiler_freed}?
    \end{minted}
    \begin{tikzpicture}[overlay,remember picture]
      \draw[thick,->] (pic cs:ownership_moved) ++(0,0.1) --
      ++(2,0)node[right]{\scriptsize Ownership of \mintinline{rust}{v} transferred to \mintinline{rust}{do_something}};
    \end{tikzpicture}
    \begin{tikzpicture}[overlay,remember picture]
      \draw[thick,->] (pic cs:ownership_compiler_freed) ++(0,0.1) --
      ++(4,0)node[right]{\scriptsize Rust compiler drops the vector \mintinline{rust}{v} here};
    \end{tikzpicture}
    \begin{tikzpicture}[overlay,remember picture]
      \draw[thick,->] (pic cs:ownership_compiler_error) ++(0,0.1) --
      ++(2,0)node[right]{\scriptsize Rust doesn't allow us to use \mintinline{rust}{v} here};
    \end{tikzpicture}
  \end{frame}

  \begin{frame}[fragile]
    \frametitle{Returning Ownership Back}
    \begin{minted}[escapeinside=??]{rust}
      pub fn main() {
        let v = vec!(1,2,3);
        let v1 = do_something(v); ?\tikzmark{return_ownership_moved}?
        println!("{:?}", v1); ?\tikzmark{return_ownership_noerror}?
      } ?\tikzmark{return_ownership_freed}?

      fn do_something(v: Vec<u64>) -> Vec<u64> {
        // Do something with v
        return v; ?\tikzmark{return_ownership}?
      }
    \end{minted}
    \begin{tikzpicture}[overlay,remember picture]
      \draw[thick,->] (pic cs:return_ownership_moved) ++(0,0.1) --
      ++(1,0)node[right]{\scriptsize Ownership of \mintinline{rust}{v} transferred to \mintinline{rust}{do_something}};
    \end{tikzpicture}
    \begin{tikzpicture}[overlay,remember picture]
      \draw[thick,->] (pic cs:return_ownership) ++(0,0.1) --
      ++(4,0)node[right]{\scriptsize Ownership of \mintinline{rust}{v} returned to calling fn};
    \end{tikzpicture}
    \begin{tikzpicture}[overlay,remember picture]
      \draw[thick,->] (pic cs:return_ownership_noerror) ++(0,0.1) --
      ++(2,0)node[right]{\scriptsize We can safely use \mintinline{rust}{v1} here};
    \end{tikzpicture}
    \begin{tikzpicture}[overlay,remember picture]
      \draw[thick,->] (pic cs:return_ownership_freed) ++(0,0.1) --
      ++(4,0)node[right]{\scriptsize Rust compiler drops the vector \mintinline{rust}{v1} here};
    \end{tikzpicture}
  \end{frame}

  \begin{frame}[fragile]
    \frametitle{Borrowing}
    \begin{minted}[escapeinside=??]{rust}
      pub fn main() {
        let v = vec!(1,2,3);
        do_something(&v); ?\tikzmark{borrow_ownership}?
        println!("{:?}", v); ?\tikzmark{borrow_ownership_noerror}?
      } ?\tikzmark{borrowed_ownership_freed}?

      fn do_something(v: &Vec<u64>) {
        // Do something with v
      }
    \end{minted}
    \begin{tikzpicture}[overlay,remember picture]
      \draw[thick,->] (pic cs:borrow_ownership) ++(0,0.1) --
      ++(1.2,0)node[right]{\scriptsize \mintinline{rust}{do_something} borrows ownership of \mintinline{rust}{v}};
    \end{tikzpicture}
    \begin{tikzpicture}[overlay,remember picture]
      \draw[thick,->] (pic cs:borrow_ownership_noerror) ++(0,0.1) --
      ++(2,0)node[right]{\scriptsize Rust doesn't complain about \mintinline{rust}{v} usage here};
    \end{tikzpicture}
    \begin{tikzpicture}[overlay,remember picture]
      \draw[thick,->] (pic cs:borrowed_ownership_freed) ++(0,0.1) --
      ++(4,0)node[right]{\scriptsize Rust compiler drops the vector \mintinline{rust}{v} here};
    \end{tikzpicture}
  \end{frame}

  \begin{frame}[fragile]
    \frametitle{Let's Mutate Things}
    \begin{minted}{rust}
      pub fn main() {
        let v = vec!(1,2,3);
        do_something(&v);
        println!("{:?}", v);
      }

      fn do_something(v: &Vec<u64>) {
        v.push(4);
      }
    \end{minted}
  \end{frame}

  \begin{frame}[fragile]
    \frametitle{Uh Oh!}
    \begin{minted}[escapeinside=??]{rust}
      pub fn main() {
        let v = vec!(1,2,3);
        do_something(&v);
        println!("{:?}", v);
      }

      fn do_something(v: &Vec<u64>) {
        v.push(4); ?\tikzmark{immutable_borrow_error}?
      }
    \end{minted}
    \begin{tikzpicture}[overlay,remember picture]
      \draw[thick,->] (pic cs:immutable_borrow_error) ++(0,0.1) --
      ++(2,0)node[right]{\scriptsize Cannot mutate immutably borrowed \mintinline{rust}{v} here};
    \end{tikzpicture}
  \end{frame}

  \begin{frame}[fragile]
    \frametitle{Everything is Mutable}
    \begin{minted}[escapeinside=??]{rust}
      pub fn main() {
        let mut v = vec!(1,2,3);
        do_something(&mut v);
        println!("{:?}", v);
      } ?\tikzmark{mutable_borrow_freed}?

      fn do_something(v: &mut Vec<u64>) {
        v.push(4);
      }
    \end{minted}
    \begin{tikzpicture}[overlay,remember picture]
      \draw[thick,->] (pic cs:mutable_borrow_freed) ++(0,0.1) --
      ++(4,0)node[right]{\scriptsize Rust compiler drops the vector \mintinline{rust}{v} here};
    \end{tikzpicture}
  \end{frame}

  \begin{frame}[fragile]
    \frametitle{One Final Note \cite{RustLangCh4}}
    \begin{minted}[escapeinside=??]{rust}
      pub fn main() {
        let mut v = vec!(1,2,3);
        let v1 = &v; //First Immutable Borrow is Fine
        let v2 = &v; //Second Immutable Borrow is Fine
        let v3 = &mut v; //Mutable and Immutable Borrows are Not Fine
        println!("{:?}", v);
      }
    \end{minted}
  \end{frame}

  \begin{frame}{Ownership and Borrowing Summary}
    \begin{itemize}
    \item Ownership once transferred, cannot be regained
    \item There is always one owner for value, which is responsible for dropping it
    \item Cannot mutate immutably borrowed content
    \item Cannot borrow both mutably and immutably at the same time
    \item Can immutably borrow any number of times
    \end{itemize}
  \end{frame}

  \begin{frame}{Why is all this necessary?}
    \begin{itemize}
    \item Eliminates common class of memory errors. For eg: Double Free Error
    \item Avoid data races by allowing only one mutable borrow
    \end{itemize}
  \end{frame}

\end{section}
