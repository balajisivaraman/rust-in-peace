\documentclass[bigger]{beamer}
\usepackage[utf8]{inputenc}
\usepackage{fixltx2e}
\usepackage{graphicx}
\usepackage{grffile}
\usepackage{longtable}
\usepackage{wrapfig}
\usepackage{rotating}
\usepackage[normalem]{ulem}
\usepackage{amsmath}
\usepackage{textcomp}
\usepackage{amssymb}
\usepackage{capt-of}
\usepackage{hyperref}
\usepackage[backend=biber,style=numeric]{biblatex}
\bibliography{rust-in-peace.bib}
\usepackage{fontspec}
\usepackage[outputdir=output]{minted}
\usepackage{tikz}
\usetikzlibrary{tikzmark}
\setmonofont{PragmataPro}
\setminted{autogobble=true,fontsize=\footnotesize}

\setbeamersize{text margin left=0.6cm,text margin right=0.6cm}
\setbeamertemplate{bibliography item}{\insertbiblabel}

\usetheme{Copenhagen}

\author{Balaji Sivaraman (\href{https://twitter.com/balajisivaraman}{@balajisivaraman})}
\date{June 28, 2018}
\title{Rust in Peace}
\institute{ThoughtWorks}
\hypersetup{
  pdfauthor={Balaji Sivaraman},
  pdftitle={Rust in Peace},
  pdfkeywords={rust-lang,systems-programming,memory-management,lifetimes},
  pdfsubject={A talk on the features of Rust that enables us to write safe, error-free, concurrent systems programs},
  pdflang={English}}

\begin{document}

\maketitle

\begin{frame}{About Me}
  \begin{itemize}
  \item Primarily worked on Java/Spring/ROR stack in ThoughtWorks, writing microservices
  \item Pure functional programming advocate in languages like Scala/Haskell/Purescript
  \item Bitten by the Rust bug last year after reading a post on how \
    it enabled Firefox's superior performance
  \item Currently on the way to transitioning from an applications \
    developer to a systems programmer, thanks primarily to Rust
  \end{itemize}
\end{frame}

\begin{section}{Introduction}
  \begin{frame}{Agenda}
    \begin{itemize}
    \item Introduction
    \item Ground Rules
    \item Systems Programming
    \item The Rust Programming Language
    \item Ownership and Borrowing
    \item Why Rust?
    \item Questions?
    \end{itemize}
  \end{frame}

  \begin{frame}{Ground Rules}
    What this talk is about?
    \begin{itemize}
    \item How Rust benefits newcomers to systems programming?
    \item What makes the Rust language unique?
    \item What benefit could I, as a web programmer, get from it?
    \end{itemize}

    What this talk is not about?
    \begin{itemize}
    \item Convince you to stop using your favourite PL and start using Rust
    \end{itemize}
  \end{frame}
\end{section}

\begin{section}{Systems Programming}

  \begin{frame}{Systems Programming}
    \begin{itemize}
    \item What is Systems Programming?
    \item Why is it different from application programming?
    \end{itemize}
  \end{frame}

  \begin{frame}{What is Systems Programming?}
    \begin{block}{From O'Reilly's Programming Rust\cite{ProgRustPreface1}:}
      \begin{quotation}
        You close your laptop. The OS detect this, suspends all the
        running programs, turns off the screen, and puts the computer to
        sleep. Later, you open the laptop: the screen and other components are
        powered up again, and program is able to pick up where it left
        off. We take this for granted. But systems programmers wrote a lot of
        code to make that happen.
      \end{quotation}
    \end{block}
  \end{frame}

  \begin{frame}{So, what is Systems Programming?}
    \begin{block}{Again from O'Reilly's Programming Rust\cite{ProgRustPreface2}:}
      \begin{quotation}
        Systems programming is \textbf{resource-constrained}
        programming. It is programming when every byte and every CPU cycle
        counts.
      \end{quotation}
    \end{block}
  \end{frame}

  \begin{frame}{What about C or C\texttt{++}?}
    \begin{itemize}
    \item Why have these languages dominated this space for 3 decades?
    \item Why is it time for a change right now?
    \end{itemize}
  \end{frame}

\end{section}

\section{Rust Programming Language}
\label{sec:rust-language}

\begin{frame}
  \centerline{
    \huge{The Rust Programming Language}
  }
\end{frame}

\begin{frame}[fragile]
  \frametitle{Can you make sense of this?}
  \begin{minted}{rust}
    pub fn main() {
      let v = vec!(1,2,3);
      let numbers: Vec<i32> = v.iter().map(|n| n * n).collect();
      println!("{:?}", numbers);
    }
  \end{minted}
\end{frame}

\begin{frame}
  \frametitle{Rust Benefits}
  \begin{itemize}
  \item High-Level Syntax (similar to Ruby or Java)
  \item Low-Level Performance (similar to C/C++)
  \end{itemize}
\end{frame}

\begin{frame}[fragile]
  \frametitle{Rust \DejaSans{❤} TDD}
  \begin{minted}{rust}
    pub fn add_one(a: u32) -> u32 {
      a + 1
    }

    #[test]
    fn test_add() {
      let result = add_one(1);
      assert_eq!(result, 2);
    }
  \end{minted}
\end{frame}

\begin{frame}[fragile]
  \frametitle{Running tests is straightforward}
  \begin{minted}{bash}
    $ cargo --test sample.rs
    $ ./sample
    running 1 test
    test test_add ... ok
    test result: ok. 1 passed; 0 failed; 0 ignored;
    0 measured; 0 filtered out
  \end{minted}
\end{frame}

\begin{frame}[fragile]
  \frametitle{Mimic OO}
  Let's define a simple Trait! \break{}
  \begin{minted}{rust}
    trait Animal {
      fn walk(&self);
    }
  \end{minted}
\end{frame}

\begin{frame}[fragile]
  \frametitle{Mimic OO}
  Let's implement it! \break{}
  \begin{minted}[fontsize=\scriptsize]{rust}
    struct Cat {
      name: String
    }

    impl Animal for Cat {
      fn walk(&self) {
        println!("{} walks like a cat", self.name);
      }
    }
  \end{minted}
\end{frame}

\begin{frame}[fragile]
  \frametitle{Mimic OO}
  Let's implement it again! \break{}
  \begin{minted}[fontsize=\scriptsize]{rust}
    struct Dog {
      name: String
    }

    impl Animal for Dog {
      fn walk(&self) {
        println!("{} walks like a dog", self.name);
      }
    }
  \end{minted}
\end{frame}

\begin{frame}[fragile]
  \frametitle{Mimic OO}
  What is the output? \break{}
  \begin{minted}{rust}
    fn main() {
      let d = Dog { name: String::from("Snuggles") };
      let c = Cat { name: String::from("Puss in Boots") };
      d.walk();
      c.walk();
    }
  \end{minted}
\end{frame}

\begin{frame}[fragile]
  \frametitle{And the output}
  \begin{minted}{shell}
    $ rustc sample.rs
    $ ./sample
    Snuggles walks like a dog
    Puss in Boots walks like a cat
  \end{minted}
\end{frame}

\setbeamertemplate{itemize/enumerate body begin}{\footnotesize}
\setbeamertemplate{itemize/enumerate subbody begin}{\scriptsize}

\begin{frame}
  \frametitle{Key Language Features}
  \begin{itemize}
  \item Functional Features
    \begin{itemize}
    \item ENums, Pattern Matching and Algebraic Data Types
    \item Lazy Iterators
    \item Functions as first class values
    \end{itemize}
  \item OO-Like Features
    \begin{itemize}
    \item Traits and Implementations
    \item Trait Bounds and Trait Objects
    \end{itemize}
  \item Rust Lang Features
    \begin{itemize}
    \item \textbf{Ownership and Borrowing}
    \item Lifetimes
    \item Unit Testing primitives as part of the core language
    \item Concurrency Primitives --- Threads, Channels, Atomic Values etc.
    \end{itemize}
  \end{itemize}
\end{frame}

\begin{section}{Ownership and Borrowing}

  \begin{frame}
    \centerline{
      \huge{Ownership and Borrowing}
    }
  \end{frame}

  \begin{frame}[fragile]
    \frametitle{A Simple Program}
    \begin{minted}{rust}
      pub fn main() {
        let v = vec!(1,2,3);
        println!("{:?}", v);
      }
    \end{minted}
  \end{frame}

  \begin{frame}[fragile]
    \frametitle{What the Rust compiler does?}
    \begin{minted}[escapeinside=??]{rust}
      pub fn main() {
        let v = vec!(1,2,3);
        println!("{:?}", v);
      } ?\tikzmark{compiler_free}?
    \end{minted}
    \begin{tikzpicture}[overlay,remember picture]
      \draw[thick,->] (pic cs:compiler_free) ++(0,0.1) --
      ++(4,0)node[right]{\scriptsize Rust compiler drops the vector \mintinline{rust}{v} here};
    \end{tikzpicture}
  \end{frame}

  \begin{frame}[fragile]
    \frametitle{Another Simple Program}
    \begin{minted}{rust}
      pub fn main() {
        let v = vec!(1,2,3);
        do_something(v);
        println!("{:?}", v);
      }

      fn do_something(v: Vec<usize>) {
        // Do something with v
      }
    \end{minted}
  \end{frame}

  \begin{frame}[fragile]
    \frametitle{What happens here?}
    \begin{minted}[escapeinside=??]{rust}
      pub fn main() {
        let v = vec!(1,2,3);
        do_something(v); ?\tikzmark{ownership_moved}?
        println!("{:?}", v); ?\tikzmark{ownership_compiler_error}?
      }

      fn do_something(v: Vec<usize>) {
        // Do something with v
      } ?\tikzmark{ownership_compiler_freed}?
    \end{minted}
    \begin{tikzpicture}[overlay,remember picture]
      \draw[thick,->] (pic cs:ownership_moved) ++(0,0.1) --
      ++(2,0)node[right]{\scriptsize Ownership of \mintinline{rust}{v} transferred to \mintinline{rust}{do_something}};
    \end{tikzpicture}
    \begin{tikzpicture}[overlay,remember picture]
      \draw[thick,->] (pic cs:ownership_compiler_freed) ++(0,0.1) --
      ++(4,0)node[right]{\scriptsize Rust compiler drops the vector \mintinline{rust}{v} here};
    \end{tikzpicture}
    \begin{tikzpicture}[overlay,remember picture]
      \draw[thick,->] (pic cs:ownership_compiler_error) ++(0,0.1) --
      ++(2,0)node[right]{\scriptsize Rust doesn't allow us to use \mintinline{rust}{v} here};
    \end{tikzpicture}
  \end{frame}

  \begin{frame}[fragile]
    \frametitle{Returning Ownership Back}
    \begin{minted}[escapeinside=??]{rust}
      pub fn main() {
        let v = vec!(1,2,3);
        let v1 = do_something(v); ?\tikzmark{return_ownership_moved}?
        println!("{:?}", v1); ?\tikzmark{return_ownership_noerror}?
      } ?\tikzmark{return_ownership_freed}?

      fn do_something(v: Vec<usize>) -> Vec<usize> {
        // Do something with v
        return v; ?\tikzmark{return_ownership}?
      }
    \end{minted}
    \begin{tikzpicture}[overlay,remember picture]
      \draw[thick,->] (pic cs:return_ownership_moved) ++(0,0.1) --
      ++(1,0)node[right]{\scriptsize Ownership of \mintinline{rust}{v} transferred to \mintinline{rust}{do_something}};
    \end{tikzpicture}
    \begin{tikzpicture}[overlay,remember picture]
      \draw[thick,->] (pic cs:return_ownership) ++(0,0.1) --
      ++(4,0)node[right]{\scriptsize Ownership of \mintinline{rust}{v} returned to calling fn};
    \end{tikzpicture}
    \begin{tikzpicture}[overlay,remember picture]
      \draw[thick,->] (pic cs:return_ownership_noerror) ++(0,0.1) --
      ++(2,0)node[right]{\scriptsize We can safely use \mintinline{rust}{v1} here};
    \end{tikzpicture}
    \begin{tikzpicture}[overlay,remember picture]
      \draw[thick,->] (pic cs:return_ownership_freed) ++(0,0.1) --
      ++(4,0)node[right]{\scriptsize Rust compiler drops the vector \mintinline{rust}{v1} here};
    \end{tikzpicture}
  \end{frame}

  \begin{frame}[fragile]
    \frametitle{Borrowing}
    \begin{minted}[escapeinside=??]{rust}
      pub fn main() {
        let v = vec!(1,2,3);
        do_something(&v); ?\tikzmark{borrow_ownership}?
        println!("{:?}", &v); ?\tikzmark{borrow_ownership_noerror}?
      } ?\tikzmark{borrowed_ownership_freed}?

      fn do_something(v: &Vec<usize>) {
        // Do something with v
      }
    \end{minted}
    \begin{tikzpicture}[overlay,remember picture]
      \draw[thick,->] (pic cs:borrow_ownership) ++(0,0.1) --
      ++(1.2,0)node[right]{\scriptsize \mintinline{rust}{do_something} borrows ownership of \mintinline{rust}{v}};
    \end{tikzpicture}
    \begin{tikzpicture}[overlay,remember picture]
      \draw[thick,->] (pic cs:borrow_ownership_noerror) ++(0,0.1) --
      ++(2,0)node[right]{\scriptsize Rust doesn't complain about \mintinline{rust}{v} usage here};
    \end{tikzpicture}
    \begin{tikzpicture}[overlay,remember picture]
      \draw[thick,->] (pic cs:borrowed_ownership_freed) ++(0,0.1) --
      ++(4,0)node[right]{\scriptsize Rust compiler drops the vector \mintinline{rust}{v} here};
    \end{tikzpicture}
  \end{frame}

  \begin{frame}[fragile]
    \frametitle{Let's Mutate Things}
    \begin{minted}{rust}
      pub fn main() {
        let v = vec!(1,2,3);
        do_something(&v);
        println!("{:?}", v);
      }

      fn do_something(v: &Vec<usize>) {
        v.push(4);
      }
    \end{minted}
  \end{frame}

  \begin{frame}[fragile]
    \frametitle{Uh Oh!}
    \begin{minted}[escapeinside=??]{rust}
      pub fn main() {
        let v = vec!(1,2,3);
        do_something(&v);
        println!("{:?}", v);
      }

      fn do_something(v: &Vec<usize>) {
        v.push(4); ?\tikzmark{immutable_borrow_error}?
      }
    \end{minted}
    \begin{tikzpicture}[overlay,remember picture]
      \draw[thick,->] (pic cs:immutable_borrow_error) ++(0,0.1) --
      ++(2,0)node[right]{\scriptsize Cannot mutate immutably borrowed \mintinline{rust}{v} here};
    \end{tikzpicture}
  \end{frame}

  \begin{frame}[fragile]
    \frametitle{Everything is Mutable}
    \begin{minted}[escapeinside=??]{rust}
      pub fn main() {
        let mut v = vec!(1,2,3);
        do_something(&mut v);
        println!("{:?}", v);
      } ?\tikzmark{mutable_borrow_freed}?

      fn do_something(v: &mut Vec<usize>) {
        v.push(4);
      }
    \end{minted}
    \begin{tikzpicture}[overlay,remember picture]
      \draw[thick,->] (pic cs:mutable_borrow_freed) ++(0,0.1) --
      ++(4,0)node[right]{\scriptsize Rust compiler drops the vector \mintinline{rust}{v} here};
    \end{tikzpicture}
  \end{frame}

  \begin{frame}[fragile]
    \frametitle{One Final Note}
    \begin{minted}[escapeinside=??]{rust}
      pub fn main() {
        let mut v = vec!(1,2,3);
        let v1 = &v; //First Immutable Borrow is Fine
        let v2 = &v; //Second Immutable Borrow is Fine
        let v3 = &mut v; //Mutable and Immutable Borrows are Not Fine
        println!("{:?}", v);
      }
    \end{minted}
  \end{frame}

  \begin{frame}{Ownership and Borrowing Summary}
    \begin{itemize}
    \item Ownership once transferred, cannot be regained
    \item Cannot mutate immutably borrowed content
    \item Cannot borrow both mutably and immutably at the same time
    \item Can immutably borrow any number of times
    \end{itemize}
  \end{frame}

\end{section}

\begin{section}{Why Rust - The Good Stuff}
  \begin{frame}
    \frametitle{Why Rust - The Good Stuff}
    \begin{itemize}
    \item The Rust Lang Book \cite{RustLangEd}
    \item Beginner Friendly Ecosystem - Rustup, Cargo, VSCode Plugin (RLS Integration) etc.
    \item Community that is accomodating of newcomers and is always glad to help
    \item Lot of scope for contributions (For eg: \href{https://github.com/rust-lang-nursery/}{Rust Lang Nursery})
    \item Growing CLI Infrastructure powered by Rust (For eg: \href{https://github.com/BurntSushi/riggrep/}{ripgrep}, \href{https://github.com/sharkdp/fd/}{fd})
    \end{itemize}
  \end{frame}
\end{section}


\begin{frame}
  \centerline{
    \huge{Questions?}
  }
\end{frame}

\begin{frame}
  \frametitle{References}
  \renewcommand*{\bibfont}{\scriptsize}
  \printbibliography
\end{frame}

\begin{frame}
  \centerline{
    \huge{Thank you!}
  }
  \centerline{
    \footnotesize{Slides source available at: \href{https://github.com/balajisivaraman/rust-in-peace}{https://github.com/balajisivaraman/rust-in-peace}}
  }
\end{frame}
\end{document}
