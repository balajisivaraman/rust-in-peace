\documentclass[bigger]{beamer}
\usepackage[utf8]{inputenc}
\usepackage{fixltx2e}
\usepackage{graphicx}
\usepackage{grffile}
\usepackage{longtable}
\usepackage{wrapfig}
\usepackage{rotating}
\usepackage[normalem]{ulem}
\usepackage{amsmath}
\usepackage{textcomp}
\usepackage{amssymb}
\usepackage{capt-of}
\usepackage{hyperref}
\usepackage{biblatex}
\usepackage{fontspec}
\usepackage{minted}
\setmonofont{PragmataPro}
\setminted{autogobble=true,fontsize=\footnotesize}
\addbibresource{rust-in-peace.bib}

\setbeamersize{text margin left=0.6cm,text margin right=0.6cm}

\usetheme{Copenhagen}

\author{Balaji Sivaraman (\href{https://twitter.com/balajisivaraman}{@balajisivaraman})}
\date{March 15, 2018}
\title{Rust in Peace}
\institute{ThoughtWorks}
\hypersetup{
  pdfauthor={Balaji Sivaraman},
  pdftitle={Rust in Peace},
  pdfkeywords={rust-lang,systems-programming,memory-management,lifetimes},
  pdfsubject={A talk on the features of Rust that enables us to write safe, error-free, concurrent systems programs},
  pdflang={English}}

\begin{document}

\maketitle

\begin{frame}{About Me}
  \begin{itemize}
  \item Primarily worked on Java/Spring/ROR stack in ThoughtWorks, writing microservices
  \item Pure functional programming advocate in languages like Scala/Haskell/Purescript
  \item Bitten by the Rust bug last year after reading a post on how \
    it enabled Firefox's superior performance
  \item Currently on the way to transitioning from an applications \
    developer to a systems programmer, thanks primarily to Rust
  \end{itemize}
\end{frame}

\begin{section}{Introduction}
  \begin{frame}{Agenda}
    \begin{itemize}
    \item Introduction
    \item Ground Rules
    \item Systems Programming
    \item The Rust Programming Language
    \item Ownership and Borrowing
    \item Why Rust?
    \item Questions?
    \end{itemize}
  \end{frame}

  \begin{frame}{Ground Rules}
    What this talk is about?
    \begin{itemize}
    \item How Rust benefits newcomers to systems programming?
    \item What makes the Rust language unique?
    \item What benefit could I, as a web programmer, get from it?
    \end{itemize}

    What this talk is not about?
    \begin{itemize}
    \item Convince you to stop using your favourite PL and start using Rust
    \end{itemize}
  \end{frame}
\end{section}


\begin{frame}
  \centerline{
    \huge{Thank you!}
  }
  \centerline{
    \footnotesize{Slides source available at: \href{https://github.com/balajisivaraman/rust-in-peace}{https://github.com/balajisivaraman/rust-in-peace}}
  }
\end{frame}
\end{document}
