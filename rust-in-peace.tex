\documentclass[bigger]{beamer}
\usepackage[utf8]{inputenc}
\usepackage{fixltx2e}
\usepackage{graphicx}
\usepackage{grffile}
\usepackage{longtable}
\usepackage{wrapfig}
\usepackage{rotating}
\usepackage[normalem]{ulem}
\usepackage{amsmath}
\usepackage{textcomp}
\usepackage{amssymb}
\usepackage{capt-of}
\usepackage{hyperref}
\usepackage[backend=biber,style=numeric]{biblatex}
\bibliography{rust-in-peace.bib}
\usepackage{fontspec}
\usepackage[outputdir=output]{minted}
\usepackage{tikz}
\usetikzlibrary{tikzmark}
\setmonofont{PragmataPro}
\setminted{autogobble=true,fontsize=\footnotesize}

\setbeamersize{text margin left=0.6cm,text margin right=0.6cm}

\usetheme{Copenhagen}

\author{Balaji Sivaraman (\href{https://twitter.com/balajisivaraman}{@balajisivaraman})}
\date{March 15, 2018}
\title{Rust in Peace}
\institute{ThoughtWorks}
\hypersetup{
  pdfauthor={Balaji Sivaraman},
  pdftitle={Rust in Peace},
  pdfkeywords={rust-lang,systems-programming,memory-management,lifetimes},
  pdfsubject={A talk on the features of Rust that enables us to write safe, error-free, concurrent systems programs},
  pdflang={English}}

\begin{document}

\maketitle

\begin{frame}{About Me}
  \begin{itemize}
  \item Primarily worked on Java/Spring/ROR stack in ThoughtWorks, writing microservices
  \item Pure functional programming advocate in languages like Scala/Haskell/Purescript
  \item Bitten by the Rust bug last year after reading a post on how \
    it enabled Firefox's superior performance
  \item Currently on the way to transitioning from an applications \
    developer to a systems programmer, thanks primarily to Rust
  \end{itemize}
\end{frame}

\begin{section}{Introduction}
  \begin{frame}{Agenda}
    \begin{itemize}
    \item Introduction
    \item Ground Rules
    \item Systems Programming
    \item Why not C or C\texttt{++}?
    \item The Rust Programming Language
    \item Getting Started
    \item Conclusion
    \item Questions?
    \end{itemize}
  \end{frame}

  \begin{frame}{Ground Rules}
    What this talk is about?
    \begin{itemize}
    \item How Rust benefits newcomers to systems programming?
    \item What modern PL design sensibilities has Rust borrowed?
    \end{itemize}

    What this talk is not?
    \begin{itemize}
    \item Fully detailed comparison between Rust \& C/C\texttt{++}
    \end{itemize}
  \end{frame}
\end{section}

\begin{section}{Systems Programming}

  \begin{frame}{Systems Programming}
    \begin{itemize}
    \item What is Systems Programming?
    \item Why is it different from application programming?
    \end{itemize}
  \end{frame}

  \begin{frame}{What is Systems Programming?}
    \begin{block}{From O'Reilly's Programming Rust \cite{ProgRustPreface1}:}
      \begin{quotation}
        You close your laptop. The OS detect this, suspends all the
        running programs, turns off the screen, and puts the computer to
        sleep. Later, you open the laptop: the screen and other components are
        powered up again, and the program is able to pick up where it left
        off. We take this for granted. But systems programmers wrote a lot of
        code to make that happen.
      \end{quotation}
    \end{block}
  \end{frame}

  \begin{frame}{So, what is Systems Programming?}
    \begin{block}{Again from O'Reilly's Programming Rust \cite{ProgRustPreface2}:}
      \begin{quotation}
        Systems programming is \textbf{resource-constrained}
        programming. It is programming when every byte and every CPU cycle
        counts.
      \end{quotation}
    \end{block}
  \end{frame}

  \begin{frame}{What about C or C\texttt{++}?}
    \begin{itemize}
    \item Why have these languages dominated this space for 3 decades?
    \item Why is it time for a change right now?
    \end{itemize}
  \end{frame}

\end{section}

\begin{section}{Rust Programming Language}
  \begin{frame}{Key Language Features}
    \begin{itemize}
    \item Functional Language Features I like in Rust
    \begin{itemize}
    \item Pattern Matching
    \item ENums similar to Algebraic Data Types
    \item Lazy Iterators
    \item Functions as first class values
    \item Error Handling Primitives using Result ADT
    \end{itemize}
    \item Rust Lang Features I Like
    \begin{itemize}
    \item \textbf{Ownership, Borrowing and Lifetimes}
    \item Unit Testing primitives as part of the core language
    \item Concurrency Primitives - Threads, Channels, Atomic Values etc.
    \end{itemize}
    \end{itemize}
  \end{frame}
\end{section}

\begin{section}{Ownership and Borrowing}
  \begin{frame}[fragile]
    \frametitle{A Simple Program}
    \begin{minted}{rust}
      pub fn main() {
        let v = vec!(1,2,3);
        println!("{:?}", v);
      }
    \end{minted}
  \end{frame}

  \begin{frame}[fragile]
    \frametitle{What the Rust compiler does?}
    \begin{minted}[escapeinside=??]{rust}
      pub fn main() {
        let v = vec!(1,2,3);
        println!("{:?}", v);
      } ?\tikzmark{compiler_free}?
    \end{minted}
    \begin{tikzpicture}[overlay,remember picture]
      \draw[thick,->] (pic cs:compiler_free) ++(0,0.1) --
      ++(4,0)node[right]{\scriptsize Rust compiler frees the vector \mintinline{rust}{v} here};
    \end{tikzpicture}
  \end{frame}

  \begin{frame}[fragile]
    \frametitle{Another Simple Program}
    \begin{minted}{rust}
      pub fn main() {
        let v = vec!(1,2,3);
        do_something(v);
        println!("{:?}", v);
      }

      fn do_something(v: Vec<usize>) {
        // Do something with v
      }
    \end{minted}
  \end{frame}

  \begin{frame}[fragile]
    \frametitle{What happens here?}
    \begin{minted}[escapeinside=??]{rust}
      pub fn main() {
        let v = vec!(1,2,3);
        do_something(v); ?\tikzmark{ownership_moved}?
        println!("{:?}", v); ?\tikzmark{ownership_compiler_error}?
      }

      fn do_something(v: Vec<usize>) {
        // Do something with v
      } ?\tikzmark{ownership_compiler_freed}?
    \end{minted}
    \begin{tikzpicture}[overlay,remember picture]
      \draw[thick,->] (pic cs:ownership_moved) ++(0,0.1) --
      ++(2,0)node[right]{\scriptsize Ownership of \mintinline{rust}{v} transferred to \mintinline{rust}{do_something}};
    \end{tikzpicture}
    \begin{tikzpicture}[overlay,remember picture]
      \draw[thick,->] (pic cs:ownership_compiler_freed) ++(0,0.1) --
      ++(4,0)node[right]{\scriptsize Rust compiler frees the vector \mintinline{rust}{v} here};
    \end{tikzpicture}
    \begin{tikzpicture}[overlay,remember picture]
      \draw[thick,->] (pic cs:ownership_compiler_error) ++(0,0.1) --
      ++(2,0)node[right]{\scriptsize Rust doesn't allow us to use \mintinline{rust}{v} here};
    \end{tikzpicture}
  \end{frame}

  \begin{frame}[fragile]
    \frametitle{Returning Ownership Back}
    \begin{minted}[escapeinside=??]{rust}
      pub fn main() {
        let v = vec!(1,2,3);
        let v1 = do_something(v); ?\tikzmark{return_ownership_moved}?
        println!("{:?}", v1); ?\tikzmark{return_ownership_noerror}?
      }

      fn do_something(v: Vec<usize>) -> Vec<usize> {
        // Do something with v
        return v; ?\tikzmark{return_ownership}?
      }
    \end{minted}
    \begin{tikzpicture}[overlay,remember picture]
      \draw[thick,->] (pic cs:return_ownership_moved) ++(0,0.1) --
      ++(1.2,0)node[right]{\scriptsize Ownership of \mintinline{rust}{v} transferred to \mintinline{rust}{do_something}};
    \end{tikzpicture}
    \begin{tikzpicture}[overlay,remember picture]
      \draw[thick,->] (pic cs:return_ownership) ++(0,0.1) --
      ++(4,0)node[right]{\scriptsize Ownership of \mintinline{rust}{v} returned to calling fn};
    \end{tikzpicture}
    \begin{tikzpicture}[overlay,remember picture]
      \draw[thick,->] (pic cs:return_ownership_noerror) ++(0,0.1) --
      ++(2,0)node[right]{\scriptsize We can safely use \mintinline{rust}{v1} here};
    \end{tikzpicture}
  \end{frame}

    \begin{frame}[fragile]
    \frametitle{Borrowing}
    \begin{minted}[escapeinside=??]{rust}
      pub fn main() {
        let v = vec!(1,2,3);
        do_something(&v); ?\tikzmark{borrow_ownership}?
        println!("{:?}", &v); ?\tikzmark{borrow_ownership_noerror}?
      }

      fn do_something(v: &Vec<usize>) {
        // Do something with v
      }
    \end{minted}
    \begin{tikzpicture}[overlay,remember picture]
      \draw[thick,->] (pic cs:borrow_ownership) ++(0,0.1) --
      ++(1.2,0)node[right]{\scriptsize \mintinline{rust}{do_something} borrows ownership of \mintinline{rust}{v}};
    \end{tikzpicture}
    \begin{tikzpicture}[overlay,remember picture]
      \draw[thick,->] (pic cs:borrow_ownership_noerror) ++(0,0.1) --
      ++(2,0)node[right]{\scriptsize Rust doesn't complain about \mintinline{rust}{v} usage here};
    \end{tikzpicture}
  \end{frame}

  \begin{frame}[fragile]
    \frametitle{Let's Mutate Things}
    \begin{minted}{rust}
      pub fn main() {
        let v = vec!(1,2,3);
        do_something(&v);
        println!("{:?}", v);
      }

      fn do_something(v: &Vec<usize>) {
        v.push(1);
      }
    \end{minted}
  \end{frame}

  \begin{frame}[fragile]
    \frametitle{Uh Oh!}
    \begin{minted}[escapeinside=??]{rust}
      pub fn main() {
        let v = vec!(1,2,3);
        do_something(&v);
        println!("{:?}", v);
      }

      fn do_something(v: &Vec<usize>) {
        v.push(1); ?\tikzmark{immutable_borrow_error}?
      }
    \end{minted}
    \begin{tikzpicture}[overlay,remember picture]
      \draw[thick,->] (pic cs:immutable_borrow_error) ++(0,0.1) --
      ++(2,0)node[right]{\scriptsize Cannot mutate immutably borrowed \mintinline{rust}{v} here};
    \end{tikzpicture}
  \end{frame}

  \begin{frame}[fragile]
    \frametitle{Let's Mutate Everything}
    \begin{minted}[escapeinside=??]{rust}
      pub fn main() {
        let mut v = vec!(1,2,3);
        do_something(&mut v);
        println!("{:?}", v);
      }

      fn do_something(v: &mut Vec<usize>) {
        v.push(1);
      }
    \end{minted}
  \end{frame}

  \begin{frame}[fragile]
    \frametitle{One Final Note}
    \begin{minted}[escapeinside=??]{rust}
      pub fn main() {
        let mut v = vec!(1,2,3);
        do_something(&mut v, &v); ?\tikzmark{mutable_immutable_borrow_error}?
        println!("{:?}", v);
      }

      fn do_something(v_mut: &mut Vec<usize>, v: &Vec<usize>) {
        // do something with v here
      }
    \end{minted}
    \begin{tikzpicture}[overlay,remember picture]
      \draw[thick,->] (pic cs:mutable_immutable_borrow_error) ++(0,0.1) --
      ++(1,0)node[right]{\scriptsize Cannot borrow \mintinline{rust}{v} both mutably and immutably};
    \end{tikzpicture}
  \end{frame}

  \begin{frame}{Ownership and Borrowing Summary}
    \begin{itemize}
    \item Ownership once transferred, cannot be regained
    \item We cannot mutate immutable borrowed content
    \item We cannot borrow both mutably and immutably at the same time
    \item We can immutably any number of times
    \end{itemize}
  \end{frame}

\end{section}

\begin{section}{Lifetimes}

  \begin{frame}
    \centerline{
      \huge{Lifetimes}
    }
  \end{frame}

  \begin{frame}[fragile]
    \frametitle{What do you expect this to do? \cite{RustLang}}
    \begin{minted}{rust}
      pub fn main() {
        let mut s1: &str = "hello";
        let mut s2: &str = "world";
        let result = do_something(&s1, &s2);
        println!("{}", result);
      }

      fn do_something(s1: &str, s2: &str) -> &str {
        //do something
        s1
      }
    \end{minted}
  \end{frame}


  \begin{frame}[fragile]
    \frametitle{Rust Befuddles Us}
    \begin{minted}[escapeinside=??]{rust}
      pub fn main() {
        let mut s1: &str = "hello";
        let mut s2: &str = "world";
        let result = do_something(&s1, &s2);
        println!("{}", result);
      }

      fn do_something(s1: &str, s2: &str) -> &str ?\tikzmark{lifetime_error}? {
        //do something
        s1
      }
    \end{minted}
    \begin{tikzpicture}[overlay,remember picture]
      \draw[thick,->] (pic cs:lifetime_error) ++(-0.5,-0.1) to[bend left]
      ++(-0.5,-1.5)node[left]{\scriptsize Rust doesn't know lifetime of returned string};
    \end{tikzpicture}
  \end{frame}

  \begin{frame}[fragile]
    \frametitle{The `Fix`}
    \begin{minted}[escapeinside=??]{rust}
      pub fn main() {
        let mut s1: &str = "hello";
        let mut s2: &str = "world";
        let result = do_something(&s1, &s2);
        println!("{}", result);
      }

      fn do_something<'a>(s1: &'a str, s2: &'a str) -> &'a str ?\tikzmark{lifetime_fix}? {
        //do something
        s1
      }
    \end{minted}
    \begin{tikzpicture}[overlay,remember picture]
      \draw[thick,->] (pic cs:lifetime_fix) ++(-0.5,-0.1) to[bend left]
      ++(-2,-1.5)node[left]{\scriptsize We say all Strings have the same life};
    \end{tikzpicture}
  \end{frame}

    \begin{frame}[fragile]
    \frametitle{Why are Lifetimes necessary? \cite{RustLang}}
    \begin{minted}[escapeinside=??]{rust}
      pub fn main() {
        let mut s1 = String::from("hello");
        let result;
        {
          let mut s2 = String::from("world");
          result = do_something(&s1, &s2);
        }
        println("{}", result);
      }

      fn do_something<'a>(s1: &'a str, s2: &'a str) -> &'a str {
        //do something
        s1
      }
    \end{minted}
  \end{frame}

  \begin{frame}[fragile]
    \frametitle{Why are Lifetimes necessary?}
    \begin{minted}[escapeinside=??]{rust}
      pub fn main() {
        let mut s1 = String::from("hello");
        let result;
        {
          let mut s2 = String::from("world");
          result = do_something(&s1, &s2);
        } ?\tikzmark{s2_lifetime_error}?
        println("{}", result);
      } ?\tikzmark{s1_lifetime_error}?

      fn do_something<'a>(s1: &'a str, s2: &'a str) -> &'a str {
        //do something
        s1
      }
    \end{minted}
    \begin{tikzpicture}[overlay,remember picture]
      \draw[thick,->] (pic cs:s2_lifetime_error) ++(0,-0.05) --
      ++(5.5,0)node[right]{\scriptsize \mintinline{rust}{s2} is dropped here};
    \end{tikzpicture}
    \begin{tikzpicture}[overlay,remember picture]
      \draw[thick,->] (pic cs:s1_lifetime_error) ++(0,-0.05) --
      ++(5.5,0)node[right]{\scriptsize \mintinline{rust}{s1} and \mintinline{rust}{result} are dropped here};
    \end{tikzpicture}
  \end{frame}

  \begin{frame}[fragile]
    \frametitle{The Lifetimes Fix}
    \begin{minted}[escapeinside=??]{rust}
      pub fn main() {
        let mut s1 = String::from("hello");
        let result;
        {
          let mut s2 = String::from("world");
          result = do_something(&s1, &s2);
          println("{}", result);
        } ?\tikzmark{s2_lifetime_fix}?
      } ?\tikzmark{s1_lifetime_fix}?

      fn do_something<'a>(s1: &'a str, s2: &'a str) -> &'a str {
        //do something
        s1
      }
    \end{minted}
    \begin{tikzpicture}[overlay,remember picture]
      \draw[thick,->] (pic cs:s2_lifetime_fix) ++(0,-0.05) --
      ++(5.5,0)node[right]{\scriptsize \mintinline{rust}{s2} and \mintinline{rust}{result} are dropped here};
    \end{tikzpicture}
    \begin{tikzpicture}[overlay,remember picture]
      \draw[thick,->] (pic cs:s1_lifetime_error) ++(0,-0.05) --
      ++(5.5,0)node[right]{\scriptsize \mintinline{rust}{s1} is dropped here};
    \end{tikzpicture}
  \end{frame}

\end{section}

\begin{section}{Why Rust - The Good Stuff}
  \begin{frame}
    \frametitle{Why Rust - The Good Stuff}
    \begin{itemize}
    \item The Rust Lang Book \cite{RustLang}
    \item Beginner Friendly Ecosystem - Rustup, Cargo, VSCode Plugin (RLS Integration) etc.
    \item Community that is accomodating of newcomers and is always glad to help
    \item Lot of scope for contributions (For eg: \href{https://github.com/rust-lang-nursery/}{Rust Lang Nursery})
    \item CLI Infrastructure powered by Rust (For eg: \href{https://github.com/BurntSushi/riggrep/}{ripgrep}, \href{https://github.com/sharkdp/fd/}{fd})
    \end{itemize}
  \end{frame}
\end{section}


\begin{frame}
  \centerline{
    \huge{Questions?}
  }
\end{frame}

\begin{frame}
  \frametitle{References}

  \printbibliography

\end{frame}

\begin{frame}
  \centerline{
    \huge{Thank you!}
  }
  \centerline{
    \footnotesize{Slides source available at: \href{https://github.com/balajisivaraman/rust-in-peace}{https://github.com/balajisivaraman/rust-in-peace}}
  }
\end{frame}
\end{document}
