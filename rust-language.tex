\section{Rust Programming Language}
\label{sec:rust-language}

\begin{frame}
  \centerline{
    \huge{The Rust Programming Language}
  }
\end{frame}

\begin{frame}[fragile]
  \frametitle{Can you make sense of this?}
  \begin{minted}{rust}
    pub fn main() {
      let v = vec!(1,2,3);
      let numbers: Vec<i32> = v.iter().map(|n| n * n).collect();
      println!("{:?}", numbers);
    }
  \end{minted}
\end{frame}

\begin{frame}
  \frametitle{Rust Benefits}
  \begin{itemize}
  \item High-Level Syntax (similar to Ruby or Java)
  \item Low-Level Performance (similar to C/C++)
  \end{itemize}
\end{frame}

\begin{frame}[fragile]
  \frametitle{Rust \DejaSans{❤} TDD}
  \begin{minted}{rust}
    pub fn add_one(a: u32) -> u32 {
      a + 1
    }

    #[test]
    fn test_add() {
      let result = add_one(1);
      assert_eq!(result, 2);
    }
  \end{minted}
\end{frame}

\begin{frame}[fragile]
  \frametitle{Running tests is straightforward}
  \begin{minted}{bash}
    $ cargo --test sample.rs
    $ ./sample
    running 1 test
    test test_add ... ok
    test result: ok. 1 passed; 0 failed; 0 ignored;
    0 measured; 0 filtered out
  \end{minted}
\end{frame}

\begin{frame}[fragile]
  \frametitle{Mimic OO}
  Let's define a simple Trait! \break{}
  \begin{minted}{rust}
    trait Animal {
      fn walk(&self);
    }
  \end{minted}
\end{frame}

\begin{frame}[fragile]
  \frametitle{Mimic OO}
  Let's implement it! \break{}
  \begin{minted}[fontsize=\scriptsize]{rust}
    struct Cat {
      name: String
    }

    impl Animal for Cat {
      fn walk(&self) {
        println!("{} walks like a cat", self.name);
      }
    }
  \end{minted}
\end{frame}

\begin{frame}[fragile]
  \frametitle{Mimic OO}
  Let's implement it again! \break{}
  \begin{minted}[fontsize=\scriptsize]{rust}
    struct Dog {
      name: String
    }

    impl Animal for Dog {
      fn walk(&self) {
        println!("{} walks like a dog", self.name);
      }
    }
  \end{minted}
\end{frame}

\begin{frame}[fragile]
  \frametitle{Mimic OO}
  What is the output? \break{}
  \begin{minted}{rust}
    fn main() {
      let d = Dog { name: String::from("Snuggles") };
      let c = Cat { name: String::from("Puss in Boots") };
      d.walk();
      c.walk();
    }
  \end{minted}
\end{frame}

\begin{frame}[fragile]
  \frametitle{And the output}
  \begin{minted}{shell}
    $ rustc sample.rs
    $ ./sample
    Snuggles walks like a dog
    Puss in Boots walks like a cat
  \end{minted}
\end{frame}

\setbeamertemplate{itemize/enumerate body begin}{\footnotesize}
\setbeamertemplate{itemize/enumerate subbody begin}{\scriptsize}

\begin{frame}
  \frametitle{Key Language Features}
  \begin{itemize}
  \item Functional Features
    \begin{itemize}
    \item ENums, Pattern Matching and Algebraic Data Types
    \item Lazy Iterators
    \item Functions as first class values
    \end{itemize}
  \item OO-Like Features
    \begin{itemize}
    \item Traits and Implementations
    \item Trait Bounds and Trait Objects
    \end{itemize}
  \item Rust Lang Features
    \begin{itemize}
    \item \textbf{Ownership and Borrowing}
    \item Lifetimes
    \item Unit Testing primitives as part of the core language
    \item Concurrency Primitives --- Threads, Channels, Atomic Values etc.
    \end{itemize}
  \end{itemize}
\end{frame}
