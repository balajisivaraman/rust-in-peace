\begin{section}{Systems Programming}

  \begin{frame}{Systems Programming}
    \begin{itemize}
    \item What is Systems Programming?
    \item Why is it different from application programming?
    \end{itemize}
  \end{frame}

  \begin{frame}{What is Systems Programming?}
    \begin{block}{From O'Reilly's Programming Rust\cite{ProgrammingRust}:}
      \begin{quotation}
        You close your laptop. The OS detect this, suspends all the
        running programs, turns off the screen, and puts the computer to
        sleep. Later, you open the laptop: the screen and other components are
        powered up again, and program is able to pick up where it left
        off. We take this for granted. But systems programmers wrote a lot of
        code to make that happen.
      \end{quotation}
    \end{block}
  \end{frame}

  \begin{frame}{So, what is Systems Programming?}
    \begin{block}{Again from O'Reilly's Programming Rust\cite{ProgrammingRust}:}
      \begin{quotation}
        Systems programming is \textbf{resource-constrained}
        programming. It is programming when every byte and every CPU cycle
        counts.
      \end{quotation}
    \end{block}
  \end{frame}

\end{section}
